\documentclass[oneside,12pt]{book}
\textwidth=16cm \oddsidemargin=0pt \evensidemargin=0pt
\usepackage[english,russian]{babel}
\usepackage{amsmath,amsthm,amssymb}
\let\amslrcorner\lrcorner
\usepackage[T2A]{fontenc}
\usepackage[utf8]{inputenc}
\usepackage{amscd}
\usepackage{oldgerm}
\usepackage[matrix,arrow,curve]{xy}
\usepackage{tikz-cd}
\usepackage{comment}
%\usepackage{xcolor}
\usepackage{hyperref}
\usepackage{mathtools}
\usepackage{extpfeil}
\usepackage{cmap}
\usepackage{quiver}
\usepackage{MnSymbol}
\usepackage{mathrsfs}
\usetikzlibrary{babel}
\definecolor{linkcolor}{HTML}{799B03} % цвет ссылок
\definecolor{urlcolor}{HTML}{799B03} % цвет гиперссылок
\hypersetup{pdfstartview=FitH,  linkcolor=linkcolor,urlcolor=urlcolor, colorlinks=true}
\vfuzz2pt % Don't report over-full v-boxes if over-edge is small
\hfuzz2pt % Don't report over-full h-boxes if over-edge is small
% THEOREMS -------------------------------------------------------
\newtheorem{thm}{Теорема}[section]
\newtheorem{exercise}[thm]{Упражнение}
\newtheorem*{thm1}{Теорема}
\newtheorem{cor}[thm]{Следствие}
\newtheorem{lem}[thm]{Лемма}
\newtheorem{zad}[thm]{Задача}
\newtheorem{prop}[thm]{Предложение}
\newtheorem{prop*}[thm]{*Предложение}
\theoremstyle{definition}
\newtheorem{dfn}[thm]{Определение}
%\newtheorem{dfn}{Определение}[section]
\newtheorem{dfn*}{*Определение}[section]
\newtheorem*{defn1}{Определение}
\theoremstyle{remark}
\newtheorem{rem}[thm]{Замечание}
\newtheorem{example}{Пример}[section]
\numberwithin{equation}{section}
\renewcommand{\proofname}{Доказательство}
\numberwithin{equation}{section}

% MATH -----------------------------------------------------------
\newcommand{\set}[1]{\left\{#1\right\}}
\newcommand{\gen}[1]{\left\langle#1\right\rangle}
\newcommand{\dpr}[2]{\left(#1\shortmid#2\right)}% Dot product
\newcommand{\eps}{\varepsilon}
\newcommand{\rank}{\mathrm{rank}}
\newcommand{\coker}{\mathrm{coker}}
\newcommand{\Hom}{\mathrm{Hom}}
\newcommand{\codim}{\mathrm{codim}}
\newcommand{\To}{\longrightarrow}
\newcommand{\0}{\varnothing}
\newcommand{\N}{\mathbb{N}}
\newcommand{\Z}{\mathbb{Z}}
\newcommand{\Q}{\mathbb{Q}}
\newcommand{\CN}{\mathbb{C}}
\newcommand{\HN}{\mathbb{H}}
\newcommand{\End}{\mathrm{End}}
\newcommand{\M}{\mathrm{M}}
\newcommand{\R}{\mathbb{R}}
\newcommand{\RN}{\mathbb{R}}
\newcommand{\tr}{\mathrm{t}}
\newcommand{\trace}{\mathrm{tr}}
\newcommand{\Aut}{\mathrm{Aut}}
\newcommand{\spec}{\mathrm{Spec}}
\newcommand{\diag}{\mathrm{diag}}
\newcommand{\GL}{\mathrm{GL}}
\newcommand{\SU}{\mathrm{SU}}
\newcommand{\OG}{\mathrm{O}}
\newcommand{\SO}{\mathrm{SO}}
\newcommand{\m}{\mathrm{M}}
\newcommand{\un}{\mathrm{U}}
\newcommand{\UG}{\mathrm{U}}
\newcommand{\re}{\operatorname{Re}}
%\DeclareMathOperator{\re}{re}
\newcommand{\im}{\operatorname{Im}}
\newcommand{\ovl}{\overline}
\newcommand{\norm}[1]{\left\lVert#1\right\rVert}
\newcommand{\mf}[1]{{\mathfrak{#1}}}
\newcommand{\id}{\operatorname{id}}
\newcommand{\sm}{\setminus\set{0}}
\newcommand{\PS}{\operatorname{P}}
\DeclareMathOperator{\noreq}{\trianglelefteq}
\newcommand{\Co}{\mathbb{C}}
\begin{document}
	% !TeX encoding = windows-1251
\begin{titlepage}
	\center{������������ ����������� � ����� ���������� ��������� \\
        ����������� ��������������� ���������� ��������������� ���������� ������� ����������� \\
        ��������� ��������������� ������������ ����������������� �����������}

\vspace{1em}

	\center{��������-�������������� ��������� \\ ������� ��������������� ����������}

\vspace{5em}

	\center \textbf{�������� ������ \\ �� ����:}
	\center \textbf{\large{�������������� ������}}
	\center \textbf{�� ���������� "������������ ���������������� ���������"}
	\center {����������� 01.03.01 ����������}

\vspace{6em}

    \begin{flushright}
        \begin{tabular}{c c}
            �������� ������� ������ ��/�~���-2021~�� & ������� �.�. \\
            \multirow{3}{23em}{������� ������������ �������� ������-�������������� ����, ������~������� ��������������� ����������} & �������� �.�.
 \\ & $\underset{\text{�������}}{\underline{\hspace{3cm}}}$
        \end{tabular}
    \end{flushright}

\vspace{\fill}

	\center{����� 2023}
\end{titlepage}
	\tableofcontents
	\newpage
	\chapter{����� ���������}

\section{�������� ���������}

\begin{dfn}
    ����� $r, n \in N$. ��� ����� $k$-������� $A$ ��������� $\mathcal{G}r_{n,r}(A)$ ��� ��������� ����� ���������� $S$ $A$-������ $A^{r+n}$, ���
    \begin{enumerate}
        \item $S$ ���������� ������ ��������� � $A^{r+n}$, �� ���� $A^{r+n} = S \oplus T$ ��� ���������� ��������� $T$ � $A^{r+n}$ (� ���������, � ���� \ref{projectivity-criteria}, $S$ -- ����������� $A$-������);
        \item $A$-������ $S$ ����� ���� $r$.
    \end{enumerate}
\end{dfn}

��� ������ �������� $\phi: \space A \To B$ ����� $S \in \mathcal{G}r_{n,r}(A)$ � $S^{B} \in \mathcal{G}r_{n,r}(B)$ (� ���� \ref{lem-isomorphism-b-modules}). ����� �������� ������������ ����������� $\mathcal{G}r_{n,r}(\phi):\;\mathcal{G}r_{n,r}(A) \To \mathcal{G}r_{n,r}(B)$ ��������� ��� $S \mapsto S^B$.

�������, ��� $\mathcal{G}r_{n,r}$ � $k$-�������.
\begin{proof}
    ����� $A \overset{\alpha}{\To} B \overset{\beta}{\To} C$ -- �������� $k$-������.\\
    �������, ��� $\mathcal{G}r_{n,r}(\beta\alpha) = \mathcal{G}r_{n,r}(\beta)\mathcal{G}r_{n,r}(\alpha)$.
    ����� $S \in \mathcal{G}r_{n,r}(A)$.
    \begin{align*}
        \left( \mathcal{G}r_{n,r}(\beta)\mathcal{G}r_{n,r}(\alpha) \right)(S) &= \mathcal{G}r_{n,r}(\beta)\left(\mathcal{G}r_{n,r}(\alpha)(S)\right) = \\
        \mathcal{G}r_{n,r}(\beta)\left( \left\langle \alpha(S) \right\rangle_B \right) &= \left\langle \beta\left( \left\langle \alpha(S) \right\rangle_B \right)  \right\rangle_C = \left\langle \left\langle (\beta\alpha)(S) \right\rangle_B \right\rangle_C = \\
        \left\langle (\beta\alpha)(S) \right\rangle_C &= \mathcal{G}r_{n,r}(\beta\alpha)(S)
    \end{align*}
    �� ��������, ��� $\mathcal{G}r_{n,r}$ � $k$-�������.
\end{proof}

	\begin{thebibliography}{99} %библиография
		\bibitem{Volochkov}
		Волочков А.А. Введение в теорию конечных групп./ / А.А.Волочков. --- Пермь, 2020.---233 стр. 58.
		\bibitem{Kolmogorov}
		Колмогоров А.Н., Фомин С.В. Элементы теории функции и функционального анализа.
		\bibitem{Milnor}
		Милнор Д., Уоллес А. Дифференциальная топология начальный курс.
		\bibitem{Hirsh}
		Хирш М. Дифференциальная топология.
		\bibitem{Jurgen}
		Jurgen J. Riemannian Geometry and Geometric.
		\bibitem{Madsen}
		Madsen. From Calculus to Cohomology
		De Rnam no cohomology and characteristic classes
		\bibitem{Lee}
		John M. Lee Introduction to Smooth Manifolds.
	\end{thebibliography}


\end{document}